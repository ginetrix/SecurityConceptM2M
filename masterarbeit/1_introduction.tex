% !TEX encoding = UTF-8 Unicode
\section{Introduction}

Security concepts can be used to capture the interacting system components, potential threats and countermeasures. 

For large information systems such concepts can become very large because of the number of the involved sub-systems/components. Interconnectivity and interdepence amongst components may increase the overall system complexity and it might be therefore difficult to detect all potential impacts \cite{branagan}.

Methods for system abstraction that address this problem already exist. The abstraction here is the creation of representation layers which only reflect relevant properties of a system and therefore provide a better level of understanding for the respective user \cite{pohl}. This for example can be achieved by different projections which reflect different granularity levels of a system displaying different levels of details \cite{thyssen2010system}.

In the security context such projections could be used to focus on the security or insecurity of certain sub-systems. Security attributes of components could thus be viewed separately and the security risk for a respective component could be derived. This might become especially useful when the security concept is incomplete or only partially available. An aggregation of security attributes based on the chosen projection will become possible. The system structure could then be used to derive security attributes for components that previously had none. Thus, potentially new information might become processable.

Aggregation methods for security attributes have already been suggested by researchers, e.g. transformation rules for security requirements by Menzel et al. \cite{Menzel2008} or aggregation rules for attack graphs by Noel et al. \cite{Noel:2004:MAG:1029208.1029225}. None of those methods take granularity levels or general system hierarchy into account whereas the goal of this thesis is to provide an approach which makes it possible for a user to select a sub-system of interest, i.e. a projection which reflects a certain granularity level and provides the corresponding security attributes. The relevant attributes as well as dependencies and possible aggregations will be shown to ensure an overall complete picture of the selected sub-system. This information can then be used to assess and improve the security level of the selected projection or its dependencies.
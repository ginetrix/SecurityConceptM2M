% !TEX encoding = UTF-8 Unicode
\section{Introduction}

Computer-related systems have theoretical and real weaknesses that could potentially be exploited by adversaries \cite{Pfleeger:2006:SC:1177321}. Computer security tries to protect these systems from theft or damage by unauthorized individuals. 

The system components, the potential threats and countermeasures as well as the interactions amongst them can be depicted in security models, such as a security concept (Section \ref{subsec:sec_concept}).

For large information systems such concepts can become very large because of the number of the involved sub-systems/components. Interconnectivity and interdependence amongst components may increase the overall system complexity and therefore the difficulty of detection of all potential impacts \cite{branagan}.

Methods for system abstraction that address this problem already exist. In this thesis system abstraction is seen as the creation of representation layers providing only relevant properties of a system. This results in a better level of understanding for the respective user \cite{pohl}. 

An exemplary approach is the creation of projections that reflect different granularity levels of a system by displaying different levels of details \cite{thyssen2010system}.

In computer security such projections could be used to focus on the security or insecurity of certain sub-systems. Security attributes of components could thus be viewed separately and the security risk for a respective component could be derived. 

As already mentioned security concepts can be used to depict security attributes of a certain computer-related system and similar to many modeling practices manual intervention is needed to some degree. This may result in incomplete or only partially available models. 

For computer security this may result in missing attributes such as threats or security goals for certain granularity levels. An approach deriving new information based on the initial model would therefore be very useful.

Aggregation methods for security attributes have already been suggested by researchers, e.g. transformation rules for security requirements by Menzel et al. \cite{Menzel2008} or aggregation rules for attack graphs by Noel et al. \cite{Noel:2004:MAG:1029208.1029225}. 

None of those methods take granularity levels or general system hierarchy into account whereas the goal of this thesis is to provide an approach which makes it possible for a user to select a sub-system of interest, i.e. a projection which reflects a certain granularity level, and provides the corresponding security attributes.

The approach presented in this thesis is not limited to a specific security attribute such as security requirements (\cite{Menzel2008}) or exploits (\cite{Noel:2004:MAG:1029208.1029225}). A complete transformation rule set for security attributes in a security concept is provided and implemented.

The relevant attributes as well as dependencies and possible aggregations will be shown to ensure an overall complete picture of the selected sub-system. This information can then be used to assess and improve the security level of the selected projection or its dependencies.

\subsection{Outline of the Thesis}

Section \ref{sec:background} provides the background material for the thesis. Computer security is briefly covered and the term \textit{Security Concept} is being introduced. Section \ref{sec:related_work} covers related areas and previous approaches of security attribute aggregation as well as granularity levels and model verification. Section \ref{sec:approach} discusses the goal of the approach and the defined transformation rules and the aggregation of security attributes. Section \ref{sec:implementation} covers the technology that was used for the implementation of the previously defined transformation rules. Section \ref{sec:conclusion} concludes by outlining the contributions and identifying avenues for future work. 
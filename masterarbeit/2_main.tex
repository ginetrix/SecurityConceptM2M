% !TEX encoding = UTF-8 Unicode
%!TEX root = thesis.tex

\newpage
\section{Background}

Prior to addressing the actual approach and implementation some concepts and terms have to be introduced. Firstly, the term \textit{security concept}, as it is used throughout the thesis, is being described. A definition of \textit{granularity levels} and system abstraction follows. Lastly, a section covers \textit{model transformations} and \textit{aggregation rules} on security attributes.

\subsection{Security}
Many different definitions of security exist. Here, a slightly adapted definition of a \textit{information security management system} (ISMS), as it is found in the ISO/IEC 27001 \cite{iso27001}, is being used.
\begin{quote}
\textit{\glqq The information security management system preserves the confidentiality, integrity and availability
of information by applying a risk management process and gives confidence to interested parties that
risks are adequately managed\grqq}    
\end{quote}

It is also added that the ISMS is integrated into the overall management structure and is vital for many of the organizational processes \cite{iso27001}.

The definition above only covers information security, which in the scope of this thesis, is insufficient. Here, we define security as the preservation of confidentiality, integrity and availability of assets, where assets can be either physical or logical.

%\subsection{Enterprise Security}
%To define the term \textit{security concept} one has to look at the architecture of enterprises to understand the interconnectivity and interdependence between services, security being one of them. 
%
%\subsubsection{Enterprise Architecture}
%
%Information systems tend to be a very complex artifacts that combine different views and requirements from various stakeholders of different backgrounds \cite{alex}. 
%
%Software, IT platforms and IT related goals in general are covered in an \textit{Information System Architecture} (ISA). ISA does not take any business-driven influences into account and is therefore insufficient when describing the complex dependencies in corporations, especially when it comes to security as described in Subsection \ref{subsec:secarch}.   
%
%\textit{Enterprise Architecture Modeling} tries to overcome such possible difficulties and combines IT related concerns with business and organizational goals and shows possible interrelationships. It therefore provides an approach for an improved understanding of complex enterprise processes \cite{earch}. The Federal Deposit Insurance Corporation (FDIC) published the results of an audit of its own implementation of E-Government principles \cite{fdicaudit} and their division of information technology in Figure \ref{fig:fdic} depicts the interrelations very well. 
%
%\begin{figure}[H]
%\centering
%\includegraphics[width=0.75\textwidth]{pictures/fdic.jpg}
%\caption{Division of Information Technology by FDIC}
%\label{fig:fdic}
%\end{figure}
%
%\subsubsection{Security Architecture}
%\label{subsec:secarch}
%
%Information Security has often been merely an afterthought in corporations \cite{ansfederal} until a concept of a \textit{Security Architecture}, published in a whitepaper by The Gartner Group \cite{kreizman}, was introduced. According to \cite{kreizman} an \textit{Enterprise Information Security Architecture} (EISA) is an essential tool for improving security processes in corporations. EISA principles stand in a direct relationship with the EA principles and should be validated against them \cite{kreizman}. To highlight this relationship security considerations during phases of the \textit{The Open Group Architecture Framework Architecture Development Method} (TOGAF ADM), which is shown in Figure \ref{fig:togaf}, will be briefly described.
%
%\begin{figure}[H]
%\centering
%\includegraphics[width=0.6\textwidth]{pictures/togaf_overview.png}
%\caption{Overview of the phases of TOGAF}
%\label{fig:togaf}
%\end{figure}
%
%Security concerns can be found throughout the TOGAF phases which hints at the overall importance of security in corporations. TOGAF combines four architecture domains, the \textit{Business Architecture}, the \textit{Data Architecture}, the \textit{Application Architecture} and the \textit{Technology Architecture}. The following section will depict security considerations of two domains to show possible interrelations.
%
%During the \textit{Business Architecture} phase in the ADM actors and handlers of the system have to be identified. Costs and potential incoveniences because of security measures have to be assessed as well. In general one can say that the impacts of security/insecurity on the business/product are being highlighted. It is tried to put an emphasis on security as early as possible to prevent costly changes in later phases in the ADM.
%
%During the \textit{Information System Architecture} phase the classification levels of processed data have to be determined and documented. Direct dependencies to the \textit{Business Architecture} are also listed, e.g. the identification of information lifespan according to business goals and regulations. 
%
%Similar relations can be found for security considerations from various phases of the ADM. This once again shows the overall presence of security and the high level of complexity within an enterprise.

\subsubsection{Common Criteria}
The overview of TOGAF showed the importance of security in corporations. The following section will present a way of modeling security concerns for an asset of interest.

Common Criteria proposes an evaluation by using a so called \textit{Security Target} (ST), a construct that encapsulates the \textit{Target of Evaluation} (TOE), threats to the TOE and countermeasures \cite{commoncriteria}. The goal of the evaluation is to show that the used countermeasures are sufficient to counter potential threats and thus implying that the TOE is sufficiently protected.

\begin{figure}[H]
\centering
\includegraphics[width=0.85\textwidth]{pictures/sectarget.png}
\caption{Overview of the Security Target contents}
\label{fig:sectarget}
\end{figure}

A description of all the contents of a ST is unnecessary here and only the key security attributes of a ST that will be used to construct a \textit{Security Concept} (Subsection \ref{subsubsec:secconc}) are being introduced.

The \textit{Security Problem Definition} defines, as the name suggests, the security problem that is being addressed. Apart from containing guidelines and assumptions it contains \textit{Threats} which are \textit{\glqq[...] adverse actions performed by a threat agent on an asset\grqq} (\cite{iso27001}, p. 66).

A \textit{Security Objective} is an abstract solution to the previously defined security problem. There exists a possibility to divide the \textit{Security Objectives} into part wise solutions, one being the \textit{Security Objectives for the TOE} and the other being the \textit{Security Objectives for the Operational Environment}. Moreover does the ST contain traces showing which objectives address which threats, guidelines and assumptions and a set showing that all threats, guidelines and assumptions are addressed by the security objective.

\textit{Security Functional Requirements} (SFR) are a more detailed translation of the previously defined \textit{Security Objective}. Despite being more detailed, SFR have to be still independent from specific technical solutions. 

Lastly, STs contain a TOE summary specification where it is stated how the TOE meets all the SFRs and how exactly those requirements are met on a technical level.

\subsubsection{Security Concept}
\label{subsubsec:secconc}

The term \textit{Security Concept}, as it is defined here, is based on the constructs introduced in the previous chapters, namely \textit{Security Architecture} and \textit{Security Target}. An overview follows. 

\textit{Assets} are the to be secured objects of interest, i.e. TOE according to Common Criteria. \textit{Assets} can be either logical or physical and can be grouped to sets, if needed.

A \textit{Security Goal} (SG) is the equivalent to the \textit{Security Objective}. A valid SG must address an \textit{Asset} and a \textit{Security Goal Class} that defines the actual purpose of the SG. In general the set of \textit{Security Goal Classes} consists of \textit{Confidentiality}, \textit{Integrity} and \textit{Availability} but can also be expanded by further classes such as \textit{Authenticity}.

\textit{Threats} serve the same purpose as proposed by Common Criteria. They are adverse actions performed by an entity against an \textit{Asset}. 

This information is all brought together in \textit{Security Requirements} that are defined in natural language and show the interrelationships between elements. A \textit{Security Goal} has to be mentioned as well as an \textit{Asset} and a \textit{Threat} against which the object of interest should be protected.

Lastly, \textit{Controls} are the technical measures that counter or minimize the \textit{Threats}.     

The Table \ref{table:secconc} depicts the relationships between all the security attributes:

\begin{sidewaystable}
\begin{tabular}{|c|c|c|c|}
\hline
\textbf{Name} & \textbf{Contains} & \textbf{Description} & \textbf{Example} \\
\hline
\multirowcell{3}{Asset} & \multirowcell{3}{-} & \multirowcell{3}{Digital or physical object of \\ interest \\ that should be secured} & \multirowcell{3}{Sensible user data} \\
& & & \\
& & & \\
\hline
\multirow{2}{*}{Security Goal Class} & \multirow{2}{*}{-} & \multirowcell{2}{Defines the purpose \\ of the Security Goal} & \multirowcell{2}{Confidentiality of sensible \\ user  data} \\
& & & \\
\hline
\multirow{2}{*}{Security Goal} & \multirow{2}{*}{Security Goal Class, Asset} & \multirowcell{2}{Defines the security \\ objective} & \multirowcell{2}{ Confidentiality of sensible \\ user data shall be protected} \\
& & & \\
\hline
\multirow{2}{*}{Threat} & \multirow{2}{*}{Asset} & \multirowcell{2}{Adverse action against an \\ Asset} & \multirowcell{2}{Eavesdropping of sensible \\ user data} \\
& & & \\
\hline
\multirowcell{4}{Security Requirement} & \multirowcell{4}{Asset, Security Goal, Threat} & \multirowcell{4}{Security Objective \\ in natural language} & \multirowcell{4}{
The Confidentiality of \\ sensible user data shall \\ be protected against \\ eavesdropping} \\
& & & \\
& & & \\
& & & \\
\hline
\multirowcell{3}{Control} & \multirowcell{3}{Threat} & \multirowcell{3}{Measure to minimize \\ or mitigate the Threat} & \multirowcell{3}{Encryption of sensible user \\ data with AES-256 \\ to prevent eavesdropping} \\
& & & \\
& & & \\

\hline
\end{tabular}
\caption{Elements of a Security Concept}
\label{table:secconc}
\end{sidewaystable}

\subsection{Modeling}
\label{subsec:interpretation}
To ensure a viable solution one has to think of a representation of real life systems. \textit{Models} can be used to achieve this by depicting the key properties and processes of a certain system. According to Ed Seidewitz \cite{seidewitz} a model is a \textit{\glqq set of statements about some system under study\grqq} with the statements being either correct or incorrect.

A system modeled using the Unified Modeling Language (UML) serves as an example. In this case such statements could be made on the relationships between classes and would only be correct if they are consistent with the actual structure of the respective system under study (SUS), i.e. the described (modeled) relationships do indeed exist. 

In our case we would try to create a model that reflects the security attributes and their interrelationships in a SUS. This interpretation of a model is key because only then the model is given a meaning \cite{seidewitz}.

A definition of a model is not enough. A \textit{metamodel} has to be clearly defined to verify whether a model is conform or not, i.e. whether a security concept instance is conform to its security concept metamodel. The following figure shows the interrelationships.

\begin{figure}[H]
\centering
\includegraphics[width=0.65\textwidth]{pictures/metamodel.png}
\caption{Relationships between Model and Metamodel}
\label{fig:metamodel}
\end{figure}

A security concept of a SUS would be modeled in a modeling language, e.g. UML which is a representation of its own metamodel. At the same time the security concept would be conform to its metamodel. This conformity, be it the security concept or the modeling language, is needed for a model to be considered valid. 

\subsubsection{Model Transformation}
\label{subsubsec:modeltrans}
The Meta Object Facility (MOF) \cite{omg2013mof} is a standard metametamodel proposed by the Object Management Group (OMG) and captures the relationships between models in a three-layered architecture consisting of M1, M2 and M3. Models (M1) are representations of systems and are expressed in a modeling language M2, e.g. UML as mentioned in the previous section, which is conform to a so called metamodel. Metamodels themselves are also expressed in a metamodeling language which is conform to a metametamodel (M3). 

These architecture levels can be found in the \textit{model transformation pattern} by Jouault et al. \cite{modeltrans} which can be seen in Figure \ref{fig:metametamodel}. 

\begin{figure}[H]
\centering
\includegraphics[width=\textwidth]{pictures/metametamodel.png}
\caption{Model transformation according to Jouault et al.}
\label{fig:metametamodel}
\end{figure}

Here a source model $Ma$ is being transformed into a target model $Mb$ using a transformation language. Both models and the transformation language are conform to their respective metamodel which is the traditional understanding of a model transformation but is unnecessarily complex for the security concept transformation.

Throughout the introduction and the background chapter the derivation of security attributes based on structural properties of a system of interest was mentioned. Given a model $M$ this derivation can be seen as alteration of $M$ and therefore as a \textit{model transformation}. The resulting model $M'$ is different to $M$, both however, are conform to the same metamodel $MM$. The transformation is being shown in Figure \ref{fig:transformation}.

\begin{figure}[H]
\centering
\includegraphics[width=0.55\textwidth]{pictures/transformation.png}
\caption{Model transformation}
\label{fig:transformation}
\end{figure}

The definition of a transformation $T$, or better a \textit{transformation rule set}, that alters a model $M$ is the main goal of this thesis.

Prior to the actual rule set definition one final concept has to be introduced. A user-selected \textit{Granularity level} serves as a second input in the model transformation.

\subsubsection{Granularity Levels}

Information systems are often complex because of the number of interconnections and interdependencies between components and therefore it might be difficult to assess potential impacts and risks of a system \cite{branagan}.

One logical goal would be to decrease the overall complexity to enable a better risk assessment. \textit{System abstraction} tries to achieve this by reducing the level of details \cite{branagan}. Branagan et al. differentiate between two possibilities, \textit{Whole-part decomposition} which decomposes a system into smaller subsystems and \textit{Distinct development perspectives} which focuses on certain parts of a system depending on the current development perspective. 

In this thesis the focus will be on the \textit{Whole-part decomposition} even though an adaption of development perspectives is certainly possible. The main difficulty would be the definition of such perspectives in the security context because they would be highly dependent on the respective security analyst. The perspectives would not be unambiguous.
  
Therefore the change of the level of detail, i.e. the granularity level, will be achieved by uniting or decomposing components of a system of interest. By decomposing a larger system into smaller subsystems one could focus on only specific security attributes and dismiss others.

A user will select a certain granularity level, i.e. a certain set of components, as an input to the transformation function $T$ as mentioned in \ref{subsubsec:modeltrans}. Together with the defined rule set a valid model $M'$ will be generated which to be considered valid has to be conform to its metamodel $MM$.

\section{Related Work}

\section{Approach}
This section presents the approach addressing the previously mentioned goal of a model transformation based on the user-selected granularity level of a system of interest. Firstly, the security concept metamodel will be thoroughly described, each element of the metamodel will be put in the security context and advantages and disadvantages of such an interpretation will be highlighted.

The second part will deal with the actual transformation rules. Model transformations will be mathematically defined and the transformation rules for each element of the model will follow. Aside from the solution possible edge cases will be presented and evaluated.

\label{sec:approach}
\subsection{Security Concept}

The following metamodel is based on the security elements mentioned in Section \ref{subsubsec:secconc}. It shows the interconnections between elements and adds restrictions. This metamodel serves as a base enabling the creation of model instances capturing the relations between components/assets of a specific SUS. It also provides a security context and the option for the user to select a granularity level. 

\begin{figure}[H]
\centering
\includegraphics[width=\textwidth]{pictures/concept_metamodel.png}
\caption{Security Concept Metamodel}
\label{fig:transformation}
\end{figure}

In the figure above all the core elements \textit{Security Goal}, \textit{Asset}, \textit{Control} and \textit{Threat} are pictured. All of these elements are part of a \textit{SecurityConcept}, which in this model will be simply identified by a name. 

SGs have a \textit{Security Goal Class} attribute which describes the purpose of each SG. The \textit{Damage Potential} attribute indicates at the importance of a Goal, i.e. how important it is to secure a certain asset. The higher the damage potential the higher the impact if the security of an asset is breached. One key aspect of this metamodel is the dependency between SGs. A SG is dependent on another SG if both belong to the same asset and have the same security goal class. These dependencies, amongst others, have to be considered during potential transformation steps (Section \ref{subsec:secgoal}). 

Each SG belongs to exactly one asset whereas an asset itself can have unlimited SGs. In this thesis both physical and virtual components can be considered an asset. Both \textit{Data} and \textit{Component} can be assets according to the metamodel. 

There will be two different types of data. For once \textit{processed data}, i.e. data that is being processed or kept in storage by a specific component. On top of that \textit{transmitted data} will be considered separately since the transmission channel itself can be seen as an asset. The resulting interconnections are shown in the following figure:

\begin{figure}[H]
\centering
\includegraphics[width=0.65\textwidth]{pictures/two_data.png}
\caption{Two different interpretations of data}
\label{fig:data}
\end{figure} 

As already mentioned data can be viewed as being processed and transmitted. Therefore an element \textit{Connection} was added. A connection is the transmission channel between two components. It must have an associated data. The processed or stored data however can be directly associated with a component. In both cases data can be viewed as an asset. There is no possibility to assign a connection as an asset, the reason being that the transmission medium itself, i.e. the cable, wire, is rarely an object of interest but more so the data which is being transmitted.

Lastly the selection of \textit{Granularity Levels} by users should be enabled. Instead of having two different input models, one security concept model and one model depicting the system structure, one can reproduce the structural properties by adding a reference to the component element.      

\begin{figure}[H]
\centering
\includegraphics[width=0.45\textwidth]{pictures/component_structure.png}
\caption{Structural information}
\label{fig:data}
\end{figure} 

Having this extra reference one can create infinitely deep structural dependencies within the model. Therefore the actual transformation will only require one model instead of two separate ones.

\subsection{Metamodel elements and their security properties}
To put the different elements of the metamodel into a security context one has to clearly define the Security Goal Classes for possible assets. Interpretations of the classes are not unambiguous and it is necessary to discuss how security attributes propagate through different abstraction layers and what kind of impact interrelated components on different layers have on each other. 

\subsubsection{Components}

Before introducing the transformation rules one has to look at the different kinds of components that can be potentially found in a model instance. Even though the model element remains the same (\textit{Component}) a distinction which is made here is necessary because the interpretation of Security Goal Classes differs depending on the component type. 

\subsubsection*{Physical Component}

Physical components are components that can be accessed physically, e.g. computers, servers, switches etc. Since those can be accessed physically and therefore be manipulated physically one has to define the Security Goal Classes accordingly.

\begin{enumerate}
\item \textbf{Confidentiality} - Will be \textit{undefined} for physical components and will only hold for data stored/processed on those components
\item \textbf{Integrity} - Ensured when the component has not been altered by an adversary in any way; can be broken by an adversary having physical access
\item \textbf{Availability} - Ensured when the component is able to carry out its designated task; can be broken by an adversary having physical access
\end{enumerate}

\subsubsection*{Virtual Component}

Virtual components cannot be directly accessed physically. Examples would be virtual machines, virtual switches etc. The classes are similar to the physical counterpart except from the breach of the respective class.

\begin{enumerate}
\item \textbf{Confidentiality} - Will be \textit{undefined} for virtual components and will only hold for data stored/processed on those components
\item \textbf{Integrity} - Ensured when the component has not been altered by an adversary in any way; can be broken remotely by an adversary, i.e. without having physical access
\item \textbf{Availability} - Ensured when the component is able to carry out its designated task; can be broken remotely by an adversary, i.e. without having physical access
\end{enumerate}

\subsubsection{Data}

Similar to the component element the data element can be interpreted in different ways.
The distinctions between the different data types are very minor but noteworthy nonetheless. We consider three types of data, data which is being stored by a component, data which is being processed by a component and data which is being transmitted between components. The common terminology is \textit{Data at Rest}, \textit{Data in Motion} and \textit{Data in Use} \cite{kanagasingham2008data}. The used terms might be new, the distinction between different states of data however can be found in publications going as far back as 1982, e.g. by Dorothy E. Denning \cite{robling1982cryptography}.

\subsubsection*{Data at Rest}

Data at Rest applies to devices that hold data \cite{kanagasingham2008data}. An example would be a database containing sensitive information. The database being the \textit{Virtual Component} and the stored data being the \textit{Data at Rest}. One could also define the server with the database as a \textit{Physical Component}.

\begin{enumerate}
\item \textbf{Confidentiality} - Ensured when the data is protected from unauthorized disclosure at rest
\item \textbf{Integrity} - Ensured when the data is protected from unauthorized modification at rest
\item \textbf{Availability} - Ensured when the data is available to authorized parties when needed, i.e. authorized parties can access and use the data when needed
\end{enumerate}

\subsubsection*{Data in Use}

Data in Use applies to data that is being processed by a component, i.e. is being used by a service running on a component. An example would be data that is being processed by an API on a server. The API being the \textit{Virtual Component} and the server being the \textit{Physical Component}. The \textit{Data in Use} would be the arriving information at the API.

\begin{enumerate}
\item \textbf{Confidentiality} - Ensured when the data is protected from unauthorized disclosure during processing
\item \textbf{Integrity} - Ensured when the data is protected from unauthorized modification during processing
\item \textbf{Availability} - Ensured when the data is available to authorized parties when needed, i.e. the data is being processed by the service/the respective service is running when needed
\end{enumerate}

\subsubsection*{Data in Motion}

Data in Motion applies to all data transmitted between components. We do not specify any protocols here to keep the definition as broad as possible.

\begin{enumerate}
\item \textbf{Confidentiality} - Ensured when the data is protected from unauthorized disclosure during transmission
\item \textbf{Integrity} - Ensured when the data is protected from unauthorized modification during transmission
\item \textbf{Availability} - Ensured when the data is available to authorized parties when needed, i.e. is not lost or intercepted during transmission 
\end{enumerate}

The different component/data types shown here should only show the different interpretations of the elements of the security concept metamodel. The interpretation of the element itself does not have an influence on the chosen metamodel element, i.e. the element for both physical and virtual components will still be \textit{Component}. The only difference can be found in data since transmitted data always belongs to a \textit{Connection}. For both processed and stored data however the metamodel element \textit{Data} will be used.

\subsubsection{Interpretation of Interconnections between Elements}

\subsection{Model Transformation}

\subsection{Transformation Rule Set for Security Goals}
\subsection{Transformation Rule Set for Threats}
\subsection{Transformation Rule Set for Controls}

\label{subsec:secgoal}
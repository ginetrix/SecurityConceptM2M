% !TEX encoding = UTF-8 Unicode
\subsection*{Abstract}
\pagestyle{empty}

Computer systems have theoretical and real weaknesses that can be exploited by adversaries \cite{Pfleeger:2006:SC:1177321}. Computer security tries to protect these systems and the contained entities from adversaries.

Security models serve the purpose of depicting the system components, potential threats and countermeasures as well as connections and used data in a comprehensible manner. 

Such components can become very complex for large information systems with a significant number of components and interrelations amongst them. This complexity can result in an increased difficulty of detection of all potential impacts \cite{branagan}.

System abstraction addresses this by creating representation layers providing only relevant properties of a system. In computer security such projections could be used to focus on the security or insecurity of certain sub-systems.

Researchers proposed aggregation methods for a variety of security attributes such as security requirements \cite{Menzel2008} or vulnerabilities/exploits \cite{Noel:2004:MAG:1029208.1029225} but were either theoretical or very limited. 

Moreover is manual intervention usually needed  when creating and maintaining security concepts  which may result in incomplete or only partially available models.

This thesis proposes an approach that is not limited to a specific security attribute such as previous work. By gathering security-related information from abstraction layers with higher and lower granularities we are able to derive new security attributes that can be used for an enhanced and more thorough security assessment of a system.
